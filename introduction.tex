%!TEX root = ./master.tex
\section{Introduction}

\section{McCulloch-Pitts Process}

Given a directed graph $G=(V,E)$ with vertex weights $\beta_i > 0$ and edge weights $\alpha_{ij} > 0$, a \emph{McCulloch-Pitts process, MPP} is an activity-based process with binary states $x \in \{0,1\}^{|V|}$ and transitions $xy$ where state $y$ is one-bit away from state $x$. If $y$ and $x$ differs in the $i$-th bit, we define the transition rate 
\begin{align*}
F_{xy} = \left[ \beta_i^{\sigma_i} \alpha_i^{x \sigma_i} \right]^{1/\tau}
\end{align*}


\begin{figure}[h]
\centering
        \begin{tikzpicture}[scale = 2,-,draw=black!50, node distance=\layersep,>=stealth]
    \tikzstyle{neuron}=[circle,fill=black!25,minimum size=20pt,inner sep=0pt];
    \tikzstyle{unit1}=[neuron, fill=red!50,thick,];
        \tikzstyle{unit2}=[neuron, fill=blue!50,thick,];
            \tikzstyle{unit3}=[neuron, fill=green!50,thick,];
 \def \radius {1cm}
 \def \n {3}
 \foreach \s in {1,...,\n}{
  \node[unit] (\s) at ({360/\n * (\s - 1) - 180}:\radius) {};
}  
   \DoubleLine{1}{2}{<-,draw=black!50}{}{->,draw=black!50}{};
   \DoubleLine{1}{3}{<-,draw=black!50}{}{->,draw=black!50}{};
    \DoubleLine{2}{3}{<-,draw=black!50}{}{->,draw=black!50}{};
      \node[unit3](k) at ({60}:\radius) {{\color{white}$x_3$}};
  \node[unit1](i) at ({180 }:\radius) {{\color{white}$x_1$}};
    \node[unit2](j) at ({300 }:\radius) {{\color{white}$x_2$}};
    \draw [->,draw=black!50, thick] (1.125) arc (20:300:1.5mm) node[pos=-0.5,left] {} (1);
\draw [->,draw=black!50, thick] (2.250) arc (-215:70:1.5mm) node[pos=-0.5,left] {} (2);
\draw [->,draw=black!50, thick] (3.15) arc (-85:195:1.5mm) node[pos=-0.5,left] {} (3);

    \node[below =.005cm of 1] {$\beta_1$};
%        \node[above left=.05cm of 3] {$\beta_3$};
            \node[right=.005cm of 2] {$\beta_2$};
    \node at (-.4,-.55) {$\alpha_{12}$};
     \node at (-.2,-.25) {$\alpha_{21}$};  
          \node at (-1.5,0.1) {$\alpha_{11}$};    
%\node at (1.255,0.8) {$w_{21}$};
%\node at (2.525,-0.1){$w_{11}$};
\end{tikzpicture}
\caption{McCulloch-Pitts process with three neurons.}
\label{fig:mpn}
\end{figure}
